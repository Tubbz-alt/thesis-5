% LaTeX preamble

% package inclusion
\usepackage[usenames,dvipsnames,svgnames]{xcolor}
\usepackage{amsthm}
\usepackage{amsmath}
\usepackage{amssymb}
\usepackage{stmaryrd}
\usepackage{setspace}
\usepackage{amstext}
\usepackage[numbers,sort&compress]{natbib}
\usepackage{prettyref}
\usepackage{verbatim}
\usepackage[greek,english]{babel}
\usepackage{mdframed}
\usepackage{graphicx}
\usepackage{enumitem}

\usepackage[top=1in, bottom=1in, left=1.5in, right=1in,includefoot,paperwidth=8.5in,paperheight=11in]{geometry}

\usepackage[pdftex]{hyperref}
\hypersetup{
    pdftitle={\Title},
    pdfauthor={\Author},
    bookmarksnumbered=true,
    bookmarksopen=true,
    bookmarksopenlevel=1,
    hidelinks,
    naturalnames=true,
    pdfstartview=Fit,
    pdfpagemode=UseOutlines,
    pdfpagelayout=TwoPageRight
}

% some cool stuff inherited from Brent
\usepackage{fancyhdr}
\lfoot[\fancyplain{}{}]{\fancyplain{}{}}
\rfoot[\fancyplain{}{}]{\fancyplain{}{}}
\cfoot[\fancyplain{}{\footnotesize\thepage}]{\fancyplain{}{\footnotesize\thepage}}
\lhead[\fancyplain{}{}]{\fancyplain{}{}}
\rhead[\fancyplain{}{}]{\fancyplain{}{}}
\ifdraft
\chead[\fancyplain{}{\textbf{DRAFT ---
    \today}}]{\fancyplain{}{\textbf{DRAFT --- \today}}}
\fi
\renewcommand{\headrulewidth}{0pt}
\setlength{\headheight}{15pt}

\newcommand{\pref}[1]{\prettyref{#1}}
\newcommand{\signature}{~ \\ \underline{\hspace{20em}}}

\newenvironment{pagecentered}{%
\vspace*{\stretch{2}}%
\begin{center}%
\begin{minipage}{.8\textwidth}%
}{%
\end{minipage}%
\end{center}%
\vspace*{\stretch{3}}\clearpage}

\newcommand{\nochapter}[1]{%
  \refstepcounter{chapter}%
  \addcontentsline{toc}{chapter}{#1}%
  \markright{#1}}

% notes to self
\ifdraft
\newcommand{\rae}[1]{\textcolor{red}{RAE: #1}}
\else
\newcommand{\rae}[1]{}
\fi

% formatting macros
\newcommand{\keyword}[1]{\textsf{\textbf{#1}}}
\newcommand{\id}[1]{\textsf{\textsl{#1}}}
\newcommand{\tick}{\text{\textquoteright}}
\newcommand{\package}[1]{\textsf{#1}}
\newcommand{\ext}[1]{\texttt{#1}}
\newcommand{\flag}[1]{\texttt{#1}}

% theorems, etc
\newtheorem{theorem}{Theorem}[chapter]
\newtheorem{lemma}[theorem]{Lemma}
\newtheorem{definition}[theorem]{Definition}
\newtheorem{notation}[theorem]{Notation}

\newrefformat{defn}{Definition \ref{#1}}

% write proposal-only stuff in blue
\ifproposal

\definecolor{proposal}{rgb}{0,0,0.5}

\newenvironment{proposal}{\color{proposal}}{}
\newenvironment{noproposal}{\comment}{\endcomment}
\else
\newenvironment{proposal}{\comment}{\endcomment}
\newenvironment{noproposal}{}{}
\fi

% that trailing / prevents the macro from gobbling up a space
\def\outsidein/{\textsc{OutsideIn}}

% more formatting
\newcommand{\bnfeq}{\ensuremath{\mathop{{:}{:}{=}}}}
\newcommand{\bnfor}{\ensuremath{\mathop{|}}}

% colorboxes
\definecolor{notyet}{rgb}{1,1,0.85}
\newmdenv[hidealllines=true,backgroundcolor=notyet,innerleftmargin=0pt,innerrightmargin=0pt,innertopmargin=-3pt,innerbottommargin=-3pt,skipabove=3pt]{notyet}
\newcommand{\notyetcolorname}{light yellow}

\definecolor{working}{rgb}{0.9,1,0.9}
\newmdenv[hidealllines=true,backgroundcolor=working,innerleftmargin=0pt,innerrightmargin=0pt,innertopmargin=-3pt,innerbottommargin=-3pt,skipabove=3pt]{working}
\newcommand{\workingcolorname}{light green}

\definecolor{noway}{rgb}{1,0.9,0.9}
\newmdenv[hidealllines=true,backgroundcolor=noway,innerleftmargin=0pt,innerrightmargin=0pt,innertopmargin=-3pt,innerbottommargin=-3pt,skipabove=3pt]{noway}
\newcommand{\nowaycolorname}{light red}


% enable numbering of subsubsections
\setcounter{secnumdepth}{3}


\newrefformat{app}{Appendix \ref{#1}}
\newrefformat{tab}{Table \ref{#1}}

%% Import ``mathb'' font
\DeclareFontFamily{U}{mathb}{\hyphenchar\font45}
\DeclareFontShape{U}{mathb}{m}{n}{
      <5> <6> <7> <8> <9> <10> gen * mathb
      <10.95> mathb10 <12> <14.4> <17.28> <20.74> <24.88> mathb12
      }{}
\DeclareSymbolFont{mathb}{U}{mathb}{m}{n}
\DeclareFontSubstitution{U}{mathb}{m}{n}

%% just so I can get this symbol
\DeclareMathSymbol{\longrightsquigarrow}{3}{mathb}{"F9}

%% and I need \leftsquigarrow, as per
%% https://tex.stackexchange.com/questions/195025/how-to-get-the-opposite-direction-of-rightsquigarrow
\makeatletter
\providecommand{\leftsquigarrow}{%
  \mathrel{\mathpalette\reflect@squig\relax}%
}
\newcommand{\reflect@squig}[2]{%
  \reflectbox{$\m@th#1\rightsquigarrow$}%
}
\makeatother


\input{ottdefns}
\newcommand{\rul}[1]{\textmd{\ottdrulename{#1}}}

\renewenvironment{ottfundefnblock}[3][]%
{\csname align*\endcsname}%
{\csname endalign*\endcsname}

\renewcommand{\ottfunclause}[2]{ #1 &= #2 \\}
\renewcommand{\ottkw}[1]{\ensuremath{\mathbf{#1}}}
