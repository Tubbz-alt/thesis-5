\begin{doublespace}

%%%%%%%%%%%%%%%%%%%%%%%%%%%%%%%%%%%%%%%%%%%%%%%%%%%%%%%%%%%%%%%%%%%%%%%%
%%%%%%%%%%%%%%%%%%%%%%%%%%%%%%%%%%%%%%%%%%%%%%%%%%%%%%%%%%%%%%%%%%%%%%%%
\begin{centering}
{\Large ABSTRACT} \\
\Title \\
\Author \\
\Advisor \\
\end{centering}
%%%%%%%%%%%%%%%%%%%%%%%%%%%%%%%%%%%%%%%%%%%%%%%%%%%%%%%%%%%%%%%%%%%%%%%%
%%%%%%%%%%%%%%%%%%%%%%%%%%%%%%%%%%%%%%%%%%%%%%%%%%%%%%%%%%%%%%%%%%%%%%%%

\vspace*{1in}

Haskell, as implemented in the Glasgow Haskell Compiler (GHC), has been adding
new type-level programming features for some time. Many of these features---gener\-al\-ized algebraic datatypes (GADTs), type families, kind
polymorphism, and promoted data\-types---have brought Haskell to the doorstep
of dependent types. Many dependently typed programs can even currently be
encoded, but often the constructions are painful.

In this dissertation, I describe Dependent Haskell, which supports full
dependent types via a backward-compatible extension to today's Haskell. An
important contribution of this work is an implementation, in GHC, of a
portion of Dependent
Haskell, with the rest to follow. The features I have implemented are already
released, in GHC 8.0.
This dissertation contains several practical examples of Dependent Haskell code,
a full description of the differences between
Dependent Haskell and today's Haskell, a novel dependently typed
lambda-calculus (called \pico/) suitable for use as an intermediate language
for compiling Dependent Haskell, and a type inference and elaboration
algorithm, \bake/, that translates Dependent Haskell to type-correct
\pico/. Full proofs of type safety of \pico/ and the soundness of \bake/ are
included in the appendix.

\end{doublespace}

%%  LocalWords:  GHC
