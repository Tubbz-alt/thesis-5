\begin{doublespace}

%%%%%%%%%%%%%%%%%%%%%%%%%%%%%%%%%%%%%%%%%%%%%%%%%%%%%%%%%%%%%%%%%%%%%%%%
%%%%%%%%%%%%%%%%%%%%%%%%%%%%%%%%%%%%%%%%%%%%%%%%%%%%%%%%%%%%%%%%%%%%%%%%
\begin{centering}
{\Large ABSTRACT} \\
\Title \\
\Author \\
Advisor: \Advisor \\
\end{centering}
%%%%%%%%%%%%%%%%%%%%%%%%%%%%%%%%%%%%%%%%%%%%%%%%%%%%%%%%%%%%%%%%%%%%%%%%
%%%%%%%%%%%%%%%%%%%%%%%%%%%%%%%%%%%%%%%%%%%%%%%%%%%%%%%%%%%%%%%%%%%%%%%%

\vspace*{1in}

Haskell, as implemented in the Glasgow Haskell Compiler (GHC) has been adding
new type-level programming features for some time. Many of these features---chiefly: generalized algebraic datatypes (GADTs), type families, kind
polymorphism, and promoted datatypes---have brought Haskell to the doorstep
of dependent types. Many dependently typed programs can even currently be
encoded, but often the constructions are painful.

In this dissertation, I describe Dependent Haskell, which supports full
dependent types via a backward-compatible extension to today's Haskell. An
important contribution of this work is an implementation, in GHC, of Dependent
Haskell. The text contains a full description of the differences between
Dependent Haskell and today's Haskell, a type-safe enhanced intermediate
language (called System FCD) supporting dependent types, and a type inference
algorithm with proved soundness and completeness results.

\end{doublespace}
